%!TEX root=../protocol.tex	% Optional

\section{Umsetzung}

% CITE: http://programmerguru.com/android-tutorial/android-restful-webservice-tutorial-how-to-call-restful-webservice-in-android-part-3/
Diese Umsetzung basiert auf GK 9.3 und dem Tutorial von programmerguru.com \cite{programmerguru-online}. Für die Entwicklung wurde Android Studio auf Windows verwendet. 

\subsection{Voraussetzungen}
Für die Umsetzung ist folgendes Vorausgesetzt:
\begin{itemize}
	\item Java-Umgebung
	\item Android SDK
	\item Ein konfiguriertes virtuelles Android-Gerät mit API-Level 23 oder höher. 
\end{itemize}

\subsection{REST-Schnittstelle}
Die Client-REST-Schnittstelle besteht aus folgenden Klassen:
\begin{itemize}
	\item \codein{java}{RESTConfig}
	\item \codein{java}{RESTExecutor}
	\item \codein{java}{RESTFailedCallback}
	\item \codein{java}{RESTSuccessCallback}
\end{itemize}

\subsubsection{RESTConfig}
Diese Klasse beschreibt die IP und den Hostname des REST-Service und dessen Endpunkte für Login und Registrierung. Es handelt sich dabei um konstante statische Felder die vom Entwickler selbst festgelegt werden.

\subsubsection{RESTFailedCallback und RESTSuccessCallback}
Diese Interfaces dienen als Callback-Interfaces für die RESTExecutor-Klasse. Da jeweils nur eine Interface-Methode deklariert wurde, kann dies mit Java Lambda-Expression verwendet werden.

\subsubsection{RESTExecutor}
Die Klasse \codein{text}{RESTExecutor} bietet statische Methoden für die Ausführung von REST-Befehlen. Intern wird die Klasse \codein{text}{AsyncHttpClient} verwendet, damit die GUI weiterhin responsiv ist. Daher müssen diese Methoden auch in der Android-Main-Loop ausgeführt werden. Für GUI-Applikationen ist dies kein Problem, da diese sowieso in der Android-Main-Loop ausgeführt werden. Unit-Tests hingegen führen Tests nicht in der Android-Main-Loop aus. Daher muss bei Unit-Tests explizit eine Task hinzugefügt werden.


\paragraph{RESTExecutor.ExecuteLoginRequest}
Diese Methode führt den REST-Request für ein Login aus. Der Endpunkt \codein{text}{<host>:port/login-request} wird angesprochen die die Parameter \codein{text}{username} und \codein{text}{password} werden übergeben. Zurückgegeben wird ein JSON-Objekt mit einem Statuscode. Je nach Statuscode wird der Callback für einen erfolgreichen Request oder der Fehler-Callback-Funktion bei einem Fehler aufgerufen.

\begin{listing}
\begin{code}{java}
public void onSuccess(int statusCode, Header[] headers, byte[] responseBody) {
	try {
		String response = new String(responseBody, "UTF-8");
		// JSON Object
		JSONObject obj = new JSONObject(response);
		
		int statusResponseCode = obj.getInt("code");
		// When the JSON response has status boolean value assigned with true
		if(statusResponseCode == 0) {
			successCallback.onSuccess();
		} else if(statusResponseCode == 1) {
			failedCallback.onFailed("User does not exist!");
		} else if(statusResponseCode == 2) {
			failedCallback.onFailed("Invalid password, try again!");
		} else { // Else display error message
		failedCallback.onFailed("Error: Unknown error code");
		}
	} catch (UnsupportedEncodingException e) {
		failedCallback.onFailed("Error Occured [Server's JSON response is not properly UTF8 encoded]!");
	} catch (JSONException e) {
		failedCallback.onFailed("Error Occured [Server's JSON response might be invalid]!");
	}
}

@Override
public void onFailure(int statusCode, Header[] headers, byte[] responseBody, Throwable error) {
	// When Http response code is '404'
	if(statusCode == 404){
		failedCallback.onFailed("Requested resource not found");
	} else if(statusCode == 500){
	failedCallback.onFailed("Something went wrong at server end");
	} else {
		failedCallback.onFailed("Unexpected Error occcured! [Most common Error: Device might not be 	connected to Internet or remote server is not up and running]");
	}
}

\end{code}
\caption{Handler für den REST-Login-Request}
\end{listing}

Je nach Fehlercode wird dann eine Nachricht zurückgegeben, die dann dem Benutzer angezeigt werden kann.

\paragraph{RESTExecutor.ExecuteRegisterRequest}
Diese Funktion arbeitet gleich wie \codein{text}{RESTExecutor.ExecuteLoginRequest} mit dem Unterschied, dass hier die Registrierung übernommen wird. Als Parameter muss Email, Nutzername und Passwort übergeben werden. Auch hier können Fehlercodes übergeben werden, jedoch zeichnen diese andere Fehler aus im Vergleich von \codein{text}{RESTExecutor.ExecuteLoginRequest}.

\begin{listing}
\begin{code}{java}
String response = new String(responseBody, "UTF-8");
// JSON Object
JSONObject obj = new JSONObject(response);

int statusResponseCode = obj.getInt("code");
// When the JSON response has status boolean value assigned with true
if(statusResponseCode == 0){
    successCallback.onSuccess();
} else if(statusResponseCode == 1) {
    failedCallback.onFailed("User already exist!");
} else if(statusResponseCode == 2) {
    failedCallback.onFailed("Failed to register, try again later!");
} else { // Else display error message
    failedCallback.onFailed("Error: Unknown error code");
}

\end{code}
\caption{Handler für den REST-Register-Request}
\end{listing}


% TODO: Cite
\subsection{GUI}
Die GUI-Klassen wurde vom Programmerguru-Tutorial übernommen und nur leicht verändert \cite{programmerguru-online}. So nimmt die Login-Activity keine E-Mail-Adresse sondern den Benutzername entgegen. Die Methode \codein{text}{invokeWS} wurde in \codein{text}{RESTExecutor} extrahiert, sodass diese mit Unit-Tests getestet werden kann. 
  

\subsection{Deployment}
Bevor das Deployment gestartet werden kann muss der Hostname des REST-Services in der Klasse \codein{text}{RESTConfig.URL_MAIN} gesetzt werden. Dies inkludiert die IP/URL inkl. Port \newline(Beispiel: \codein{java}{URL_MAIN = "http://192.168.0.50:3000"}).

Als nächstes muss der REST-Service selbst gestartet werden. Dies ist im Protokoll von GK 9.3 beschrieben.


% CITE https://stackoverflow.com/questions/4974568/how-do-i-launch-the-android-emulator-from-the-command-line
Als nächstes muss ein Emulator gestartet werden. Dies kann mit \codein{text}{emulator @name-of-your-emulator} erzielt werden (wo ``name-of-your-emulator'' ist der Name des Emulators). Will man wissen welche virtuelle Geräte vorhanden sind kann man dies mit \codein{text}{emulator -list-avds} auflisten. \cite{stackoverflow-launch}

Die Emulator-Binary ist in \codein{text}{${ANDROID_SDK}/tools/emulator} zu finden.

Wurde der Emulator gestartet und das Android-Betriebssystem vollständig initialisiert, so kann nun das Projekt gestartet werden. Folgende Aktionen sind möglich:
\begin{itemize}
	\item \codein{text}{gradlew build} - Baut das Projekt
	\item \codein{text}{gradlew installDebug} - Erstellt eine Debug-APK-Datei, welches auf einem virtuellen Android-Gerät installiert wird. Dabei muss ein virtuelles Android-Gerät bereits laufen.
	\item \codein{text}{gradlew connectedDebugAndroidTest} - Erstellt eine Debug-APK-Datei und lässt die Tests auf einem virtuellen Android-Gerät ausgeführt wird. Dabei muss eine virtuelles Android-Gerät bereits laufen.
\end{itemize}

Nach dem Starten des Emulators und dem Ausführen des Befehls \codein{text}{gradlew installDebug} sollte die App im Launcher des virtuellen Android-Gerät vorhanden und ausführbar sein.



